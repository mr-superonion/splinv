\documentclass[twocolumn]{aastex63}
\emergencystretch=1.em

\usepackage{amsmath,amsthm,amsfonts,amssymb,bm}
\usepackage{physics,commath}

\usepackage{listings}
\hypersetup{breaklinks}
\usepackage{color}
\usepackage{graphicx}
\usepackage{multirow}

\usepackage{textcomp}
\usepackage{epstopdf}
\usepackage{natbib}
\usepackage{algorithm,algcompatible}
\usepackage[-]{callouts}

\makeatletter
\newcommand*\bigcdot{\mathpalette\bigcdot@{.5}}
\newcommand*\bigcdot@[2]{\mathbin{\vcenter{\hbox{\scalebox{#2}{$\m@th#1\bullet$}}}}}
\newcommand{\argmax}{\mathop{\rm arg~max}\limits}
\newcommand{\argmin}{\mathop{\rm arg~min}\limits}

\DeclareMathOperator{\arccosh}{arccosh}
\newcommand \redColor{\color{red}}

\renewcommand\labelenumi{(\roman{enumi})}
\renewcommand\theenumi\labelenumi

\algnewcommand\INPUT{\item[\textbf{Input:}]}
\algnewcommand\OUTPUT{\item[\textbf{Output:}]}

\newcommand{\includezraphics}[1]{
    \begin{annotate}{
        \includegraphics[width=0.95\textwidth]{#1}
    }{1.0}
    \arrow { -3., -2}{ -3., 2}
    \note {-3.,2.5}{z}
    \end{annotate}
}

\makeatother

\begin{document}
\title{Cluster Detection from three-dimensional Lensing Mass Map}
% \author{Xiangchong Li}
% \affiliation{Department of Physics, University of Tokyo, Tokyo 113-0033, Japan}
% \affiliation{Kavli Institute for the Physics and Mathematics of the Universe (WPI),\\
% University of Tokyo, Kashiwa 277-8583, Japan}
% \author{Masamune Oguri}
% \affiliation{Department of Physics, University of Tokyo, Tokyo 113-0033, Japan}
% \affiliation{Kavli Institute for the Physics and Mathematics of the Universe
% (WPI),\\
% University of Tokyo, Kashiwa 277-8583, Japan}
% \affiliation{Research Center for the Early Universe, University of Tokyo, Tokyo
% 113-0033, Japan}
% \author{Wentao Luo}
% \affiliation{Kavli Institute for the Physics and Mathematics of the Universe
% (WPI),\\
% University of Tokyo, Kashiwa 277-8583, Japan}
% \email{xiangchong.li@ipmu.jp}

\begin{abstract}
A new method is developed to reconstruct high-resoluion three-dimensional mass
maps from photometric weak-lensing shear measurements. The three-dimensional
mass map is modeled as a summation of the NFW basis atoms, which have
two-dimensional multi-scale NFW surface density profiles on the transverse
plane and one-dimensional Dirac delta functions in the line of sight direction.
The adaptive lasso algorithm is applied to find a sparse reconstruction of the
mass maps.
We study the performance of three-dimensional cluster detection from the
reconstructed mass map with simulations that applies shear distortions from
isolated halo to HSC-like galaxy shapes with realistic photometric redshift
uncertainties .
Our findings are summarized as follows:
1) The lasso reconstructed suffers from a smear of structure in the line of
sight direction even in the absence of shape noise, and in contrast, the
adaptive lasso algorithm efficiently removes the line of sight smear.
2) The algorithm is able to detect halo with minimal mass limits of $10^{14.0}
M_{\odot}/h$, $10^{14.7} M_{\odot}/h$, $10^{15.0} M_{\odot}/h$ for the low
($z<0.3$), median ($0.3\leq z< 0.6$) and high ($0.6\leq z< 0.85$) redshifts,
respectively, with an average false detection of 0.022/deg$^2$.
3) The estimated redshift of the halos detected from the reconstructed mass
maps are lower than the true redshift by about $0.03$ for halos at low
redshifts ($z\leq 0.4$). The relative redshift bias is below $0.5\%$ for halos
at $0.4<z\leq 0.85$. The standard deviation of the redshift estimation is $0.092$.
\end{abstract}

\section{Introduction}

\begin{figure*}[!t] \includegraphics[width=1.\textwidth]{nfwlet-atom-2D.pdf}
    \caption{The smoothed pixelized basis atoms. The upper row shows the basis
        atoms in Fourier space, and the lower row shows the basis atoms in Real
        space.  The leftmost column is the point mass atom, and the other
        columns are the multi-scale NFW atoms.  The smoothing kernel is
        Gaussian with a standard dev iation of $1.5$ pixels.
    } \label{fig_atoms2D}
\end{figure*}

\begin{figure}
 \includegraphics[width=0.45\textwidth]{nfwlet-atom-1D.pdf}
    \caption{The $1$-D slices for smoothed pixelized basis atoms at $x=0$. The
        corresponding $2$-D profiles are shown in Figure \ref{fig_atoms2D}.
    }
 \label{fig_atoms1D}
\end{figure}

Weak lensing refers to the phenomenon that light from distant galaxies is
weakly yet coherently distorted by the intervening inhomogeneous density
distribution along the line of sight due to gravity's influence.  As a result,
the shapes of the background galaxies are distorted and the information of the
foreground mass distribution is imprinted to the background galaxy images. Weak
lensing offers a direct probe into the mass distribution, including both
visible matter and invisible dark matter, in our universe \citep[see][for
recent reviews]{revKilbinger15,revRachel17}.
Several large-scale surveys focus on studying the weak-lensing effect at a high
precision level (e.g., HSC \citep{HSC1-data}, KIDS \citep{KIDS13}, DES
\citep{DES05}, LSST \citep{LSSTScienceBook}, Euclid \citep{Euclid2011}, NGRST
\citep{WFIRST15}).

% The primary goal of most weak-lensing surveys is to constrain the cosmology
% model through the $2$-point correlations. The studies include galaxy$-$galaxy
% lensing, which cross-correlating the shear field ($\gamma$) with the positions
% of foreground galaxies
% \citep{gglens-GAMA-Han2014,gglens-BossCFHTMore2015,gglens-DES1}, and cosmic
% shear which auto-correlates the shear measurements
% \citep{cosmicShearRealKids450,cosmicShear-DES1,cosmicShear_HSC1_Chiaki2019,cosmicShear_HSC1_Hamana2019}.
% Since the shear is directly related to the foreground matter distribution as
% shown in eq. (\ref{eq-intro-delta2shear}), Galaxy$-$galaxy lensing probes into
% the correlation between the matter field and galaxy field. On the other hand,
% cosmic shear probes into the auto-correlation of matter field.

The reconstruction of density map from shear measurements has received
considerable interest since
we can constrain the cosmology model using the peak counts, which are
identified from the weak lensing mass map
\citep{WL-massMap-peakcounts-Jain2000,WL-massMap-peakcountsAna-Fan2010,WL-massMap-peakcountsFM-Lin2016}.
Moreover, we can directly detect massive clusters through
identifying high signal-to-noise (SNR) peaks on the mass map without any
reference to the mass-to-light ratio
\citep{WL-massMap-clusDet-Schneider1996,WL-massMap-clusDet-Hamana2004}.

The two-dimensional ($2$-D) density map reconstruction which recovers an
integration of projected mass along the line of sight has been well studied
within the community
\citep{massMap-KS1993,WL-massMap-Glimpse2D-Lanusse2016,sparseBaysianMassMap-Price2020}
and applied to large-scale surveys
\citep{HSC1-massMaps,massMapDES-Chang2018,DES-SV-massMap-sparsity}.
Cluster detection from $2$-D mass maps has been widely studied and applied to
these large-scale weak-lensing surveys
\citep{WL-massMap-clusDet-CFHT-Shan2012,WL-massMap-clusDet-HSC-Miyazaki2018,WL-massMap-clusDet-HSC-Hamana2020};
however, cross matching with some (e.g., optically selected) cluster catalog is
needed to assign the redshift and extract its physical parameter (such as mass)
for each cluster selected from $2$-D mass map.

On the other hand, we can directly reconstruct $3$-D mass maps taking advantage
of photometric redshift (photo-$z$) information
\citep{LSS-massMap-Wiener-Simon2009,WL-massMap-VanderPlas2011}; however, these
methods either do not have enough spatial resolution to identify clusters, or
they suffer from smears along the line of sight direction, which need to be
overcome for practical searches of clusters in 3D mass maps.  Alternatively,
\citet{WL-clusDet-Hennawi2005} propose to perform a maximum-likelihood
detection of clusters, by convolving tomographic shear measurements with $3$-D
filters that match the tangential shears induced by multi-scale NFW halos,
without fully reconstruct the $3$-D mass maps.

In this paper, we develop a novel method for high-resolution $3$-D mass map
reconstruction that are free from line of sight smears.  The $3$-D
mass map is modeled as a summation of the NFW \citep{halo-NFW1997ApJ} basis
atoms, which have $2$-D multi-scale NFW surface density profiles on the
transverse plane and one-dimensional Dirac delta functions in the line of sight
direction. The adaptive lasso algorithm \citep{AdaLASSO-Zou2006} is applied to
find a sparse reconstruction of the mass maps.

We study the performance of cluster detection from the reconstructed mass maps
using simulations applying shear distortions from isolated halos to HSC-like
galaxy shapes with realistic photo-$z$ uncertainties.


This paper is organized as follows.
In Section \ref{sec_Method}, we propose the new method for $3$-D density map
reconstruction.
In Section \ref{sec_Test}, we study the cluster detection from the
reconstructed mass map using isolated halo simulations with the HSC
observational condition.
In Section \ref{sec_Sum}, we summarize and discuss the future development of
the method.


\section{$3$-D Mass Map Reconstruction}
\label{sec_Method}
\input{massmap_method.tex}


\section{Cluster Detection}
\label{sec_Test}

This Section simulates weak-lensing shear fields induced by a group of NFW
halos with various halo masses and redshifts. The shear fields are used to
distort the HSC mock shape catalogs with different realizations of the HSC-like
shape measurement error and photo-$z$ uncertainty (Section \ref{subsec_Sims}).

Then, we test our algorithm using the simulations with different setups of the
regularization parameter. We also compare the results of our algorithm, which
uses the NFW dictionary (Section \ref{subsec_test-nfw}), with the point mass
dictionary (Section \ref{subsec_test-pm}).

\subsection{Simulations}
\label{subsec_Sims}
\begin{figure}
 \centering
 \includegraphics[width=0.5\textwidth]{shapeMeasurementError-HSCY1.pdf}
 \caption{The solid lines show the histograms of the HSC-like shape measurement
     error (including both from shape noise and photon noise) on the first
     component of shear ($g_1$) for galaxies (blue lines) and smoothed pixels
     (orange lines). The dashed lines are the best-fit Gaussian distributions
     to the corresponding histograms.
    }
 \label{fig_noiseHistogram}
\end{figure}

The $\Lambda$CDM cosmology used in this paper is from the best-fit result of
the final full-mission Planck observation of the cosmic microwave background
(CMB) with $H_0=67.4 ~\rm{km~s^{-1} Mpc^{-1}}$ $\Omega_M=0.315$,
$\Omega_\Lambda=0.685$, $\sigma_8=0.811$, $n_s=0.965$
\citep{cmb-Planck2018-Cosmology}.

We sample halos in a two-dimensional redshift-mass plane. The redshift-mass
plane is evenly divided into eight redshift bins and eight mass bins. We
randomly shift the input halo redshifts and halo masses from the bins' centers
by a small amount. In the simulation, we set the non-linear over-density
($\Delta_{\rm{vir}}$) to $200$, and $M_{200}$ refers to the virial mass. The
concentration of the NFW halo is set to as a function of the halo's virial mass
($M_{200}$) and redshift ($z_{h}$) according to
\citet{c-M_Magneticum-Ragagnin2019}
\begin{equation}
c_{h}=6.02\times\left(\frac{M_{200}}{10^{13} M_{\odot}}\right)^{-0.12}
\left(\frac{1.47}{1.+z_h}\right)^{0.16}.
\end{equation}
The halos are truncated at the virial radius.
The weak-lensing shear fields of these NFW halos are simulated according to
\citet{haloModel-TJ2003-3pt}. The shear distortions are applied to one hundred
realizations of galaxy catalogs with the HSC-like shape measurement error and
photo-$z$ uncertainty.

\begin{figure}
 \centering
 \includegraphics[width=0.5\textwidth]{noise_std_map_pix.pdf}
 \caption{The standard deviation pixel map of the HSC-like shape measurement
     error for the fifth source galaxy bin ($0.69 \leq z < 0.80 $).
        } \label{fig_noistdmap}
\end{figure}

The mock galaxy catalogs are generated using the HSC S16A shape catalog
\citep{HSC1-catalog}.  We use the galaxies in a one square degree region at the
center of tract 9347 \citep{HSC1-data}. The average galaxy number density in
this region is $22.94~\rm{arcmin}^{-2}$. The galaxies' positions are randomized to
distribute homogeneously  in the one-square degree stamp statistically. We
randomly assign its redshift for each galaxy following the MLZ photo-$z$
probability distribution function \citep{HSC1-photoz}.

By randomly rotating the galaxies in the shape catalog, we simulate the
HSC-like shape estimation errors with different realizations.  The histogram of
the first component of the HSC-like shape estimation error on galaxy level is
shown in Figure \ref{fig_noiseHistogram}.  The corresponding histogram of the
shape measurement error on the pixel level after the smoothing and pixelization is
also shown in \ref{fig_noiseHistogram}. The standard deviation map of the
noise is demonstrated in Figure \ref{fig_noistdmap}. As demonstrated in Figure
\ref{fig_noiseHistogram}, even though the shape measurement error on the galaxy
level does not fully follow Gaussian distribution, the error is well described
by Gaussian distribution after the smoothing and pixelization.


\subsection{NFW atoms}
\label{subsec_test-nfw}
\begin{figure*}[!t]
\begin{minipage}[c]{1.0\columnwidth}
\includegraphics[width=0.95\textwidth]{pixel_histograms_halo17_NFW_lbd35.pdf}
    \includezraphics{delta-1-7-pz-wn-NFW-falsepeakproblem.pdf}
    \centering
    \small NFW: $\lambda=3.5$
\end{minipage}
\begin{minipage}[c]{1.0\columnwidth}
\includegraphics[width=0.95\textwidth]{pixel_histograms_halo17_NFW_lbd50.pdf}
    \includezraphics{delta-1-7-pz-wn-NFW_3-falsepeakproblem.pdf}
    \centering
    \small
    NFW: $\lambda=5.0$
\end{minipage}
\caption{The lower panels show the density maps reconstructed from the mock
    galaxy shape catalog with the NFW dictionary. The upper panels show the
    pixels' number histograms.  The penalization parameters are $\lambda=3.5$
    (left) and $\lambda=5.0$ (right).  The input halo mass is
    $M_{200}=10^{15.02} ~h^{-1}M_{\odot}$, and its redshift is $z=0.164$.  The
    vertical direction is the line of sight direction. The boxes' lower
    boundaries and upper boundaries of correspond to $z=0.01$ and $z=0.85$,
    respectively.
    } \label{fig_NFW3D}
\end{figure*}

In this subsection, we test the performance of our algorithm with the default
setup that models the matter density field with multi-scale NFW atoms. The
dictionary is constructed with three frames of different NFW scale radii in the
comoving coordinate: $0.12~h^{-1}$ Mpc, $0.24~h^{-1}$ Mpc, and $0.36~h^{-1}$
Mpc.  The truncation radii are set to four times the comoving scale radii for
the atoms in the dictionary (concentration equals four) . We note that each
frame of our dictionary fixes the scale radius in the comoving coordinates;
therefore, the NFW atoms have different angular radii in different lens
redshift bins.

We test the algorithm with different regularization parameters ($\lambda$) for
the preliminary lasso estimation, which are $3.5$, $4.0$, and $5.0$. The
corresponding regularization parameters for the final adaptive lasso
estimations are set to $\lambda_{\rm{ada}}=\lambda^{\tau+1}$.  Here, we note
that both the preliminary lasso estimation and the final adaptive lasso
estimation select the pixels with the SNRs greater than $\lambda$ in each
gradient descent iteration and estimate the density in the selected pixels.
While the final adaptive lasso estimation further enhances the growth of the
pixels with preliminary estimations greater than $\lambda$.

This paper does not go beyond the resolution limit defined by the Gaussian
smoothing kernel with a standard deviation of $1.\arcmin 5$ and the $1\arcmin$
pixel scale as discussed in Section \ref{subsec_method_smoothing} and Section
\ref{subsec_method_pixel}, respectively.  Therefore, we smooth the
reconstructed density with the same Gaussian kernel in each lens redshift
plane.

Figure \ref{fig_NFW3D} shows the $3$-D density maps reconstructed with
different penalization parameters for a halo with $M_{200}=10^{15.02}
~h^{-1}M_{\odot}$ at redshift $0.164$. Also, the pixels' histograms are shown
in Figure \ref{fig_NFW3D}. From these plots, we conclude that the adaptive
lasso algorithm sets most of the reconstructed pixels to zero and only keeps
the modes strongly related to the data. Moreover, the reconstructed density maps
do not suffer from the line of sight smearing. After the reconstruction for
each simulation, we identify the peaks on the sparse density map.

\begin{figure*}
 \centering
 \includegraphics[width=1.0\textwidth]{peak_histograms_NFW.pdf}
 \caption{The number per square degree histograms of detected peak values from
     all of the simulations.  The solid red steps result from reconstructions
     with the NFW dictionary penalized with different regularization
     parameters: $\lambda=3.5,4.0,5.0$. The dashed blue steps are the
     corresponding results of the reconstructions from $1000$ realizations of
     pure noise catalogs. The gray lines are the best-fit Gaussian distributions
     to the noises' peak histograms.
    }\label{fig_peakHist}
\end{figure*}

Following \citet{WL-massMap-Glimpse2D-Lanusse2016}, we normalize the detected
peaks in the $l$-th ($l=1...20$) lens redshift plane to account for the peak
amplitude difference due to the difference in the norm of the lensing kernels
for different redshift bins:
\begin{equation}
\delta^{\rm{n}}_{\rm{peak}}(\vec{\theta},z_l)=\delta_{\rm{peak}}(\vec{\theta},z_l)/\mathcal{R}_{l}^{\frac{1}{2}},
\end{equation}
where the normalization matrix is defined as
\begin{equation}
\mathcal{R}_{l}=\sum_s K^2(z_l,z_s).
\end{equation}
The solid steps in Figure \ref{fig_peakHist} show the histograms of the
normalized peaks with different penalization parameters. Also, we simulate
$1000$ realizations of pure noise catalogs and perform the reconstructions on
these noise catalogs to study the noise properties. The dashed steps in Figure
\ref{fig_peakHist} show the histograms of normalized peaks detected from the
pure noise catalogs. The solid lines in Figure \ref{fig_peakHist} show the
best-fit Gaussian distributions to the noise peaks' histograms.

Figure \ref{fig_peakHist} tells that the densities of peaks (including both
true and false peaks) are suppressed as the penalization parameter $\lambda$
increases. Moreover, we find the standard deviation of noise peaks slightly
decreases as $\lambda$ increases. As a result, for a higher detection threshold
($\lambda=5.0$), we find a clearer peak number excess for mass maps
reconstructed from mock catalogs comparing with the noise peak histogram .

The $2$-D histogram, stacked from all of the simulations, for the offsets of
the detected peak positions from the input halos' positions is shown in the
left panel of Figure \ref{fig_detoffsets}. We see a clustering of peaks close
to the input halo's position on the stacked position histogram.  For each stamp
simulation, we find the positive peak closest to the input position (in the
pixel unit). If the closest peak lay inside the region denoted with the dashed
box in the left panel of Figure \ref{fig_detoffsets}, we take it as a true peak
detection of the input halo. Other identified peaks, which include both positive
and negative peaks, are taken as false detections.

\begin{figure*}
 \centering
 \includegraphics[width=1.0\textwidth]{peak_scatters_NFW_lbd35.pdf}
 \caption{The left panel shows the stacked $2$-D distribution of the deviations
     of detected peak positions from the centers of the corresponding input
     halos. The $x$-axis is for the deviated distance in the transverse plane,
     and the $y$-axis is for the deviation of the redshift. For each
     simulation, the positive peak inside the dashed black box with the minimal
     offset (in the pixel unit) from the input halo's position is taken as a
     true detection. The right panel focuses on the deviation of detected peaks
     in the line of sight direction. The $x$-axis is the input halo redshifts,
     and the $y$-axis is the redshift of the detected peak. The `$\cross$'
     denotes the average redshift of detected peaks for each halo over
     different noise realizations, and the error-bars are the uncertainties of
     the average redshifts. The deep gray area is for the relative redshift
     bias less than $0.05$, and the light gray area is for the relative
     redshift bias less than $0.5$. These results in this figure are based on
     the NFW dictionary with $\lambda=3.5$.
     } \label{fig_detoffsets}
\end{figure*}

The right panel of Figure \ref{fig_detoffsets} shows the average redshift of
true detections for each halo. The estimated redshifts are lower than the true
redshifts by about $0.03$ for halos in the low-redshift range ($z\leq
0.4$).  For halos at $0.4<z\leq 0.85$, the relative redshift bias is below
$0.5\%$.

With the intent to suppress false detections, we select peaks with values
greater than an ad-hoc threshold as candidates of galaxy clusters following
\citet{WL-massMap-clusDet-HSC-Miyazaki2018}.  The threshold is set to a few
times the standard deviation of the noise peaks. We use different detection
thresholds ($1.5\sigma$ and $3.0\sigma$) to detect galaxy clusters from the
mass maps reconstructed with $\lambda=3.5,4.0,5.0$. The left and middle columns
of Figure \ref{fig_detFalsRateNFW} show the detection rates for halos in the
(mass, redshift) planes with detection thresholds set to $1.5\sigma$ and
$3.0\sigma$ of the noise peaks' distributions, respectively.  The right column
of Figure \ref{fig_detFalsRateNFW} shows the corresponding numbers of false
detections per square degree as functions of detection thresholds. The first,
second, and third rows of Figure \ref{fig_detFalsRateNFW} correspond to the
$lambda=3.5,4.0,5.0$, respectively.

Figure \ref{fig_detFalsRateNFW} tells that the false peak density is suppressed
as the detection threshold increases. Also, the detection rate of halo
significantly decreases. We decide to set the detection threshold to
$1.5\sigma$ and set the penalization parameter $\lambda$ to $5.0$ since such a
setup suppresses the false detection to $0.022$ while keeping a good halo
detection rate. In summary, The algorithm is able to detect halo with minimal
mass limits of $10^{14.0} M_{\odot}/h$, $10^{14.7} M_{\odot}/h$, $10^{15.0}
M_{\odot}/h$ for the low ($z<0.3$), median ($0.3\leq z< 0.6$) and high
($0.6\leq z< 0.85$) redshifts, respectively.

\begin{figure*}
 \centering
 \includegraphics[width=1.0\textwidth]{detfalse_threshold_NFW_lbd35.pdf}
 \includegraphics[width=1.0\textwidth]{detfalse_threshold_NFW_lbd40.pdf}
 \includegraphics[width=1.0\textwidth]{detfalse_threshold_NFW_lbd50.pdf}
 \caption{The detection rates and false peak densities for different
     penalization parameters and detection thresholds. The first, second, and
     third rows correspond to the results with $\lambda=3.5,4.0,5.0$,
     respectively. The left and middle columns are the halo detection rates for
     detection thresholds equal $1.5\sigma$ and $3.0\sigma$, respectively. The
     right column shows the density of false peaks as a function of detection
     threshold.
    } \label{fig_detFalsRateNFW}
\end{figure*}

According to the detection rate measured from the simulation, we predict the
number density of detected cluster using the halo mass function of
\citet{haloMass-Tinker2008}. We use HMF \citep{hmf-Murray2013}, an open-source
package, to calculate the halo mass function.  The predicted halo detection
number density for the setup ($\lambda=5$, $1.5\sigma$ detection threshold) is
shown in Figure \ref{fig_detNum}. The expected cluster number density in total
is $0.49$ $\deg^{-2}$, which corresponds to $78.4$ clusters for the first year
HSC shear catalog \citep{HSC1-catalog} with a survey area of $\sim 160 \deg^2$.
The expected number of detections is similar to the number of $2$-D cluster
detections on the first year HSC shape catalog \citep{HSC1-massMap-cluster}.

\begin{figure}
 \centering
 \includegraphics[width=0.5\textwidth]{detNum_zm.pdf}
 \caption{The expected number density of detected clusters per square degree as
     a function of halo's virial mass and redshift. The number density in total
     is $0.49~\deg^{-2}$.
     } \label{fig_detNum}
\end{figure}


\subsection{Point mass atoms}
\label{subsec_test-pm}

In this subsection, we show the results for the setup, which substitutes the
default NFW dictionary with the point mass dictionary, to compare with the
default setup.  The regularization parameter for the preliminary lasso is set
to $3.5$ and $5.0$.  Inspired by \citet{structureAdaLasso-Pramanik2020}, which
propose to incorporate external group information into different adaptive lasso
penalization weights by setting the penalization weights for projection
coefficients in the same group to the average of the adaptive weights in this
group, we smooth the preliminary lasso estimation in each lens redshift bin
with a top-hat filter of comoving scale $0.25~h^{-1}$ Mpc. The smoothed
preliminary lasso estimation is denoted as $\hat{x}^{\rm{ls}}_{\rm{sm}}$, and
the penalization weights are set to $1/\abs{\hat{x}^{\rm{ls}}_{\rm{sm}}}^\tau$.
The regularization parameter for the adaptive lasso is set to
$\lambda_{\rm{als}}=\lambda_{\rm{ls}}$.

As we have done on the NFW dictionary's reconstruction, we smooth the
reconstructed density maps with the Gaussian kernel (scale radius equals
$1.\arcmin 5$) in each lens redshift plane.

Figure \ref{fig_detFalsRatePM} shows the detection rate (left panels) and the
number of false peaks per square degree (right panel) for each setup of halo
mass and redshift.
This figure includes results of different penalization parameters:
$\lambda=3.5$ (upper panel) and $\lambda=5.0$ (lower panel). The right panels
of Figure \ref{fig_NFWvsPM} show the $3$-D density maps reconstructed with
different penalization parameters for a halo with $M_{200}=10^{15.02}
~h^{-1}M_{\odot}$ at redshift $0.164$.
Comparing with the results of the NFW dictionary, we find that the number of
false peaks in the mass map reconstructed with the point mass dictionary is
much larger than the NFW dictionary.

As demonstrated in the right panels of Figure \ref{fig_NFWvsPM}, the mass
reconstructions with the point mass dictionary tend to assign masses to
several different redshift bins in the neighboring region of the halo center.
While the NFW dictionary produces consistent mass reconstructions.
Since the NFW atoms' profiles are more consistent with the mass profiles of the
input halos than the profile of a point mass, there are less false structures
reconstructed with the NFW dictionary.




\section{Summary}
\label{sec_Sum}

We have developed a novel method to reconstruct high-resolution $3$-D density
contrast maps from weak-lensing shear measurements and photometric redshift
estimations.  Our method models $3$-D density contrast maps as summations of
NFW atoms with different comoving radii.  With the prior assumption that the
clumpy masses sparsely distribute in the $3$-D space, the density map is
reconstructed using the adaptive lasso algorithm \citep{AdaLASSO-Zou2006}.
We find that the lasso algorithm's solution suffers from a smear of structure
in the line of sight direction even in the absence of galaxy shape noise, and
the adaptive lasso algorithm efficiently removes the line of sight smear of
structure.

We study the performance of cluster detection from the reconstructed $3$-D mass
maps using simulations that apply shear distortions from isolated NFW halos to
HSC-like shapes and photo-$z$ uncertainties.  We conclude that the algorithm is
able to detect halo with minimal mass limits of $10^{14.0} M_{\odot}/h$,
$10^{14.7} M_{\odot}/h$, $10^{15.0} M_{\odot}/h$ for the low ($z<0.3$), median
($0.3\leq z< 0.6$) and high ($0.6\leq z< 0.85$) redshifts, respectively, with an
average false detection of 0.022 deg$^{-2}$. The estimated redshifts of the halos
detected from the reconstructed mass maps are lower than the true redshift by
about $0.03$ for halos at low redshifts ($z\leq 0.4$).  The relative redshift
bias is below $0.5\%$ for halos at $0.4<z\leq 0.85$. The standard deviation of
the redshift estimation is $0.092$.

We will apply the $3$-D mass map reconstruction to the shear measurements of
the HSC survey \citep[e.g.,][]{HSC1-catalog,FPFSHSC1-Li2020} to perform galaxy
cluster detection in our future work.

\section*{Acknowledgements}
XL thank Yin Li and Jiaxin Han for useful discussions.

XL was supported by Global Science Graduate Course (GSGC) program of
University of Tokyo and JSPS KAKENHI (JP19J22222).
\bibliographystyle{aasjournal}
\bibliography{citation}

\end{document}
