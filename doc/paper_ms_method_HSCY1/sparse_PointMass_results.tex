
We substitute the default NFW dictionary with the point mass dictionary and
reconstruct the mass map from the mock galaxy shape catalog, to compare with
the default setup.

The regularization parameter ($\lambda$) for the preliminary lasso is set to
$3.5$ and $5.0$.  Inspired by \citet{structureAdaLasso-Pramanik2020}, which
propose to incorporate external group information into different adaptive lasso
penalization weights by setting the penalization weights for projection
coefficients in the same group to the average of the adaptive weights in this
group, we smooth the preliminary lasso estimation in each lens redshift bin
with a top-hat filter of comoving diameter: $0.25~h^{-1}$ Mpc. The smoothed
preliminary lasso estimation is denoted as $\hat{x}^{\rm{ls}}_{\rm{sm}}$, and
the penalization weights are set to
$\hat{w}=1/\abs{\hat{x}^{\rm{ls}}_{\rm{sm}}}^\tau$.

As demonstrated in Figure \ref{fig_PM3D}, the mass reconstructions with the
point mass dictionary tend to assign masses to several different redshift bins
in the neighboring region of the halo center.  In contrast, as demonstrated in
Figure \ref{fig_NFW3D}, the NFW dictionary manages to perform consistent mass
reconstructions. The problem of the point mass dictionary presumably originates
from the fact that the profile of the point mass atom in the transverse plane
is much more compact than the profile of the input halos, especially at low
redshift.

\begin{figure*}
\centering
\begin{minipage}[c]{1.0\columnwidth}
    \includezraphics{delta-1-7-pz-wn-PM-falsepeakproblem.pdf}
    \centering
    \small Point Mass: $\lambda=3.5$
\end{minipage}
\begin{minipage}[c]{1.0\columnwidth}
    \includezraphics{delta-1-7-pz-wn-PM_3-falsepeakproblem.pdf}
    \centering
    \small Point Mass: $\lambda=5.0$
\end{minipage}
\caption{The density maps reconstructed from the mock galaxy shape catalog with
    the point mass dictionary. The penalization parameters are $\lambda=3.5$
    (left) and $\lambda=5.0$ (right).  The input halo mass is
    $M_{200}=10^{15.02} ~h^{-1}M_{\odot}$, and its redshift is $z=0.164$.  The
    vertical direction is the line of sight direction. The lower boundaries and
    the upper boundaries of the boxes correspond to $z=0.01$ and $z=0.85$,
    respectively.
    } \label{fig_PM3D}
\end{figure*}
