\newcommand \figPathM{/home/xiangchong/.local/code/massMap_Private/doc/paper_ms_method_HSCY1/}
\newcommand \figPathMF{/home/xiangchong/.local/code/massMap_Private/doc/fig/}
\newcommand{\argmax}{\mathop{\rm arg~max}\limits}
\newcommand{\argmin}{\mathop{\rm arg~min}\limits}

\subsection{Model Dictionary and Regularization}
\begin{frame}{Goal}
\begin{equation} \notag
    \gamma=\mathbf{T} \delta+\rm{noise},
\end{equation}
where $\mathbf{T}$ includes both physical signal and systematics
\begin{columns}[t]
\begin{column}{0.5\textwidth}
{\color{blue}\rule{\linewidth}{4pt}
Physical Signal}\\
\begin{equation}\notag
\kappa(\vec{\theta},z_s)=\frac{3H_0^2\Omega_M}{2 c^2} \int_0^{\chi_s} d\chi_l \frac{\chi_l \chi_{sl}}{\chi_s}
\frac{\delta(\vec{\theta},z_l)}{a(\chi_l)},
\end{equation}
\begin{equation}\notag
\gamma_L(\vec{\theta},z_s) = \int  d^2 \theta' D(\vec{\theta}-\vec{\theta'}) \kappa(\vec{\theta'},z_s),
\end{equation}
where
\begin{equation}\notag
D(\vec{\theta})=-\frac{1}{\pi}(\theta_1-i\theta_2)^{-2}.
\end{equation}
\end{column}
\hfill
\begin{column}{0.5\textwidth}
{\color{red}\rule{\linewidth}{4pt}
Systematics}
\begin{itemize}
    \item photo-$z$ uncertainty;
    \item Masking;
    \item Smoothing in transverse plane;
    \item Pixelization.
\end{itemize}
\end{column}
\end{columns}
\alert{The Goal is to inverse the matrix $\mathbf{T}$ and estimate
$3$-D density contrast field: $\delta(\vec{\theta},z_l)$.}
\end{frame}

\begin{frame}{Prior Information}
Since the $3$-D inversion problem is an \alert{ill-posed problem}, we have to add \alert{prior information}.
\begin{columns}[t]
\begin{column}{0.5\textwidth}
\begin{alertblock}{Model Dictionary}
\begin{equation} \notag
\delta(\vec{\theta},z)=\Sigma_{i} \Phi_i(\vec{\theta},z) x_i,
\end{equation}
where $\Phi_i$ is the model basis, and $x_i$ is the projected modes.
e.g.,
\begin{enumerate}
    \item Point Mass;
    \item Fourier Space (sine, cosine);
    \item Starlets.
\end{enumerate}
\end{alertblock}
\end{column}

\begin{column}{0.5\textwidth}
\begin{alertblock}{Regularization}
\begin{equation} \notag
\hat{x}=\argmin_{x} \left\{ \frac{1}{2}\norm{\Sigma^{-\frac{1}{2}}(\gamma-
\mathbf{T} \Phi x)}_2^2+ \lambda \norm{x}_p^p \right\},
\end{equation}
\begin{equation}\notag
\norm{x}_p^p=(\sum_{i} |x|^p_i).
\end{equation}
e.g.,
\begin{enumerate}
    \item $p=2$, \alert{Ridge regression} (Wiener Filter);
    \item $p=1$, \alert{Lasso} (Sparse);
    \item $p=0$, \alert{Best subset} (sparsest but unsolvable).
\end{enumerate}
\end{alertblock}
\end{column}
\end{columns}
Choose a \alert{model dictionary} and a \alert{regularization
(Prior distribution of $x$)}.
\end{frame}


\subsection{Line of Sight Smearing}
% \begin{frame}{Line of sight smearing}
% \begin{figure}
% \includegraphics[height=0.6 \textwidth]{\figPathMF Richard_Massey_HST.png}
% \end{figure}
% Richard Massey (HST)
% \end{frame}

\begin{frame}{Wiener Filter in S16A}
\begin{figure}
\includegraphics[height=0.6 \textwidth]{\figPathMF Masamune_Oguri_HSC.png}
\end{figure}
Masamune Oguri, 2018 (HSC), Wiener Filter
\end{frame}

\begin{frame}{NFW Model dictionary}
\begin{columns}
\begin{column}{0.4\textwidth}
\begin{figure}
\includegraphics[width=1.0\textwidth]{\figPathM nfwlet-atom-1D.pdf}
\end{figure}
\begin{itemize}
    \item pixel size:  $1$ arcmin
    \item Gaussian Smoothing: $1.5$ arcmin
\end{itemize}
\end{column}
\begin{column}{0.6\textwidth}
\begin{alertblock}{NFW Atoms}
\begin{eqnarray}\notag
\phi_\alpha(\vec{r}) =&\frac{f }{2 \pi \theta_\alpha^2 }
F(|\vec{\theta}|/\theta_\alpha) \delta_D(z),\\
&  (\alpha=1..N)\notag
\end{eqnarray}
\begin{equation}\notag
F(x)=
\begin{cases}
-\frac{\sqrt{c^2-x^2}}{(1-x^2)(1+c)} + \frac{\rm{arccosh}
\left(\frac{x^2+c}{x(1+c)}\right)}{(1-x^2)^{3/2}}  & (x<1),\\
\frac{\sqrt{c^2-1}}{3(1+c)} (1+\frac{1}{c+1}) & (x=1),\\
-\frac{\sqrt{c^2-x^2}}{(1-x^2)(1+c)} +
\frac{\arccos\left(\frac{x^2+c}{x(1+c)}\right)}{(x^2-1)^{3/2}} & (1<x\leq c),\\
0& (x>c).
\end{cases}
\end{equation}
Takada \& Jain (2003)
\end{alertblock}
\end{column}
\end{columns}
\end{frame}

\begin{frame}{Adaptive Lasso (regularization/Prior on $x$)}
\alert{Approximation to $l^0$ regularization with two steps of
$l^1$ lasso estimation}\\
Zou (2007)
\begin{alertblock}{first step: lasso}
\begin{equation}\notag
\hat{x}=\argmin_{x} \left\{ \frac{1}{2} \norm{\Sigma^{-\frac{1}{2}}(\gamma-\mathbf{A}x)}^2_2 +
\lambda \norm{x}^1_1\right\}.
\end{equation}
\end{alertblock}
\begin{equation}
\hat{w}= \frac{1}{\abs{\hat{x'}^{\rm{lasso}}}^\tau},
\end{equation}
hyper-parameter: $\tau=2$.
\begin{alertblock}{second step: weighted lasso}
\begin{equation}\notag
\hat{x}=\argmin_{x} \left\{ \frac{1}{2} \norm{\Sigma^{-\frac{1}{2}}(\gamma-\mathbf{A}x)}^2_2 +
\hat{w}\lambda_{\rm{ada}} \norm{x}^1_1\right\}.
\end{equation}
\end{alertblock}
\end{frame}


\begin{frame}{Results for Noiseless Simulation}
\begin{itemize}
    \item HSC (s16) number densitty;
    \item HSC (s16) redshifts(best estimation);
    \item No noise; no photo-$z$ uncertainty.
\end{itemize}
\begin{figure}
\centering
\includegraphics[width=1.0\textwidth]{\figPathMF results_noiseless.png}
\end{figure}
\end{frame}


\begin{frame}{Correlated Lensing Kernels}
\begin{figure}
 \centering
 \includegraphics[width=1.0\textwidth]{\figPathM lensing_kernel.pdf}
\end{figure}
\alert{The lensing kernels for two neighbouring lens redshift bins
are too correlated (The shapes of them are similar). Therefore, it
is difficult to distinguish in which bins the mass is actually
located.}
\end{frame}


\begin{frame}{Results for HSC mock catalogs}
\begin{itemize}
    \item HSC (s16) number densitty;
    \item HSC (s16) redshifts(best estimation);
    \item HSC (s16) shape noise; HSC photo-$z$ uncertainty.
\end{itemize}
\begin{figure}
\centering
\includegraphics[width=1.0\textwidth]{\figPathMF results_noisy.png}
\end{figure}
\end{frame}

\begin{frame}{Redshift estimation}
    We simulation halos with different mass at different redshift; each halo with $100$ noise realizations. The following figures show the \alert{offsets} of detected peaks from the input positions.
\begin{figure}
\centering
\includegraphics[width=1.0\textwidth]{\figPathM peak_scatters_NFW_lbd35.pdf}
\end{figure}
\end{frame}

\begin{frame}{Peak Histogram}
$1000$ pure noise realizations with HSC shape noise and photoz
uncertainty.
\begin{figure}
\centering
\includegraphics[width=1.0\textwidth]{\figPathM peak_histograms_NFW.pdf}
\end{figure}
\end{frame}



