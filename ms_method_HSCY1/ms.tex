\documentclass[twocolumn]{aastex62}
\emergencystretch=1.em 
\newcommand \figPath{../fig/}
\newcommand \redColor{\color{red}}
\renewcommand\labelenumi{(\roman{enumi})}
\renewcommand\theenumi\labelenumi
%\newcommand{\jcap}{JCAP}


\usepackage{amsmath,amsthm,amsfonts,amssymb,bm}
\usepackage{listings}
\usepackage{multirow}
\usepackage{color}

%\usepackage{subfigure}
\usepackage{textcomp}
\usepackage{epstopdf}
\usepackage{natbib}
\hypersetup{breaklinks}


\begin{document}
\title{mass map reconstruction with sparsity method}
%\author{Xiangchong Li}
%\affiliation{Department of Physics, University of Tokyo, Tokyo 113-0033, Japan}
%\affiliation{Kavli Institute for the Physics and Mathematics of the Universe (WPI),\\
%University of Tokyo, Kashiwa 277-8583, Japan}
%\author{Masamune Oguri}
%\affiliation{Department of Physics, University of Tokyo, Tokyo 113-0033, Japan}
%\affiliation{Kavli Institute for the Physics and Mathematics of the Universe (WPI),\\
%University of Tokyo, Kashiwa 277-8583, Japan}
%\affiliation{Research Center for the Early Universe, University of Tokyo, Tokyo 113-0033, Japan}
%\author{Nobuhiko Katayama}
%\affiliation{Kavli Institute for the Physics and Mathematics of the Universe (WPI),\\
%University of Tokyo, Kashiwa 277-8583, Japan}
%\author{Wentao Luo}
%\affiliation{Kavli Institute for the Physics and Mathematics of the Universe (WPI),\\
%University of Tokyo, Kashiwa 277-8583, Japan}
%\author{Wenting Wang}
%\affiliation{Kavli Institute for the Physics and Mathematics of the Universe (WPI),\\
%University of Tokyo, Kashiwa 277-8583, Japan}
%\author{Jiaxin Han}
%\affiliation{Department of Astronomy, School of Physics and Astronomy, Shanghai Jiao Tong University, Shanghai, 200240, China}
%\affiliation{Kavli Institute for the Physics and Mathematics of the Universe (WPI),\\
%University of Tokyo, Kashiwa 277-8583, Japan}
%\author{HSC Collaboration}
%\noaffiliation
%\email{xiangchong.li@ipmu.jp}




\begin{abstract}

\end{abstract}

\section{Introduction}


\section{The Method}
\label{sec:kappaMap}

\subsection{Kaiser-Squires reconstruction}
\label{subsec:kappaMap-KS}
Subsequently, the shear field is converted to the convergence field via \citep{massMap-KS1993}
\begin{equation}\label{eq:KS_real}
\kappa(\vec{\theta})=\frac{1}{\pi} \int d^2\theta' \frac{\gamma_t(\vec{\theta'}|\vec{\theta})}{|\vec{\theta}-\vec{\theta'}|^2},
\end{equation}
where $\hat{\gamma}_t(\vec{\theta'}|\vec{\theta})$ is a tangential shear at position $\vec{\theta'}$ computed with respect to the reference position $\vec{\theta}$. The shear-to-convergence relationship is a convolution in two dimensional angular plane. Such convolution is computed in Fourier space with the Fast Fourier Transform (FFT) to reduce the computational time. Shear fields are padded with zero beyond their boundary to avoid the periodic boundary condition assumed in FFT. The complex shear field is denoted as $\gamma(\vec{\theta})=\gamma_1(\vec{\theta})+i\gamma_2(\vec{\theta})$. The Fourier transform of complex shear field and complex kappa field is denoted as $\tilde{\gamma}(\vec{l})$ and $\tilde{\kappa}(\vec{l})$, respectively. Eq. (\ref{eq:KS_real}) can be expressed in Fourier space
\begin{equation}
\tilde{\kappa}(\vec{l})=\pi^{-1}\tilde{\gamma}(\vec{l})\tilde{D}^{*}(\vec{l}) ~~ for ~~\vec{l}\neq\vec{0},
\end{equation}
where $\tilde{D}(\vec{l})$ is the Fourier transform of the convolution kernel in eq. (\ref{eq:KS_real})
\begin{equation}
\tilde{D}(\vec{l})=\pi \frac{l_1^2-l_2^2+2il_1l_2}{|\vec{l}|^2}.
\end{equation}
\break
Then the mass map in configuration space $(\kappa(\vec{\theta}))$ is reconstructed by inverse Fourier transforming $\tilde{\kappa} (\vec{l})$.
Note that the real part of the reconstructed mass map is referred as an E-mode mass map, whereas the imaginary part of the reconstructed mass map is referred as an B-mode mass map which is used to check for certain types of residual systematics in weak lensing measurements.


The relation between $\gamma$ and $\kappa$ is 
\begin{equation}
\gamma(\vec{\theta})=\int D(\vec{\theta}-\vec{\theta}') \kappa(\vec{\theta}') d^2\theta.
\end{equation}
Denote $\gamma$ and $\kappa$ as vectors $\vec{\gamma}$ and $\vec{\kappa}$, respectively. $\vec{\kappa}$ is the mass map to be reconstructed on equally spaced coordinates. $\vec{\gamma}$ is the shear observed from non-equally spaced galaxies. Then the operation of convolution can be denoted as a matrix multiplication
\begin{equation}
\vec{\gamma}=\textbf{D} \vec{\kappa} + \vec{n}.
\end{equation}
And $\vec{\kappa}$ can be determined by minimize
\begin{equation}
\chi^2=(\vec{\gamma}-\textbf{D} \vec{\kappa})^{T}(\vec{\gamma}-\textbf{D} \vec{\kappa}).
\end{equation}
We require 
\begin{equation}
dim\left(\vec{\kappa}\right) \leq dim \left( \vec{\gamma} \right)
\end{equation}

Find an orthogonal vector space and decompose $\vec{\kappa}$ onto the orthogonal space, the decomposition factor are denoted as $\vec{\alpha}$, where
\begin{equation}
\begin{split}\\
\vec{\alpha}&= \bold{\Phi}\vec{\kappa}\\
\vec{\kappa}&=\bold{\Phi}^{-1} \vec{\alpha}
\end{split}
\end{equation} 
If $\vec{\alpha}$ only has a small number of nonzero coefficients,
\begin{equation}
\vec{\gamma}=\textbf{D} \bold{\Phi}^{-1}\vec{\alpha}+\vec{n}
\end{equation}

\begin{equation}
\chi^2=||(\vec{\gamma}-\textbf{D} \bold{\Phi}^{-1}\vec{\alpha})||^2_{2} + \lambda||\vec{\alpha}||_{1}
\end{equation}
\begin{equation}
\begin{split}
||\vec{v}||_{1}&=\sum_i^{N} |v_i|\\
||\vec{v}||^2_{2}&=\sum_i^{N} v_i^2
\end{split}
\end{equation}
\bibliographystyle{aasjournal}
\bibliography{weak_lensing,lensCat,cosmoSim,cmbObs,gglens,shearCor,massMap}



\end{document}
